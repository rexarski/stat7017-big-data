%
% This is a borrowed LaTeX template file for lecture notes for CS267,
% Applications of Parallel Computing, UCBerkeley EECS Department.
% Now being used for CMU's 10725 Fall 2012 Optimization course
% taught by Geoff Gordon and Ryan Tibshirani.  When preparing 
% LaTeX notes for this class, please use this template.
%
% To familiarize yourself with this template, the body contains
% some examples of its use.  Look them over.  Then you can
% run LaTeX on this file.  After you have LaTeXed this file then
% you can look over the result either by printing it out with
% dvips or using xdvi. "pdflatex template.tex" should also work.
%

\documentclass[twoside]{article}
\setlength{\oddsidemargin}{0.25 in}
\setlength{\evensidemargin}{-0.25 in}
\setlength{\topmargin}{-0.6 in}
\setlength{\textwidth}{6.5 in}
\setlength{\textheight}{8.5 in}
\setlength{\headsep}{0.75 in}
\setlength{\parindent}{0 in}
\setlength{\parskip}{0.1 in}

%
% ADD PACKAGES here:
%

\usepackage{amsmath,amsfonts,graphicx,soul}

%
% The following commands set up the lecnum (lecture number)
% counter and make various numbering schemes work relative
% to the lecture number.
%
\newcounter{lecnum}
\renewcommand{\thepage}{\thelecnum-\arabic{page}}
\renewcommand{\thesection}{\thelecnum.\arabic{section}}
\renewcommand{\theequation}{\thelecnum.\arabic{equation}}
\renewcommand{\thefigure}{\thelecnum.\arabic{figure}}
\renewcommand{\thetable}{\thelecnum.\arabic{table}}

\newcommand*\conj[1]{\bar{#1}}
\newcommand*\mean[1]{\bar{#1}}
\newcommand{\indep}{\rotatebox[origin=c]{90}{$\models$}}
%
% The following macro is used to generate the header.
%
\newcommand{\lecture}[4]{
   \pagestyle{myheadings}
   \thispagestyle{plain}
   \newpage
   \setcounter{lecnum}{#1}
   \setcounter{page}{1}
   \noindent
   \begin{center}
   \framebox{
      \vbox{\vspace{2mm}
    \hbox to 6.28in { {\bf STAT7017 Big Data Statistics
	\hfill Semester 2 2018} }
       \vspace{4mm}
       \hbox to 6.28in { {\Large \hfill Lecture #1: #2  \hfill} }
       \vspace{2mm}
       \hbox to 6.28in { {\it Lecturer: #3 \hfill Scribes: #4} }
      \vspace{2mm}}
   }
   \end{center}
   \markboth{Lecture #1: #2}{Lecture #1: #2}

   {\bf Note}: {\it LaTeX template courtesy of UC Berkeley EECS dept.}

   {\bf Disclaimer}: {\it These notes have not been subjected to the
   usual scrutiny reserved for formal publications.  They may be distributed
   outside this class only with the permission of the Instructor.}
   \vspace*{4mm}
}
%
% Convention for citations is authors' initials followed by the year.
% For example, to cite a paper by Leighton and Maggs you would type
% \cite{LM89}, and to cite a paper by Strassen you would type \cite{S69}.
% (To avoid bibliography problems, for now we redefine the \cite command.)
% Also commands that create a suitable format for the reference list.
\renewcommand{\cite}[1]{[#1]}
\def\beginrefs{\begin{list}%
        {[\arabic{equation}]}{\usecounter{equation}
         \setlength{\leftmargin}{2.0truecm}\setlength{\labelsep}{0.4truecm}%
         \setlength{\labelwidth}{1.6truecm}}}
\def\endrefs{\end{list}}
\def\bibentry#1{\item[\hbox{[#1]}]}

%Use this command for a figure; it puts a figure in wherever you want it.
%usage: \fig{NUMBER}{SPACE-IN-INCHES}{CAPTION}
\newcommand{\fig}[3]{
			\vspace{#2}
			\begin{center}
			Figure \thelecnum.#1:~#3
			\end{center}
	}
% Use these for theorems, lemmas, proofs, etc.
\newtheorem{theorem}{Theorem}[lecnum]
\newtheorem{lemma}[theorem]{Lemma}
\newtheorem{proposition}[theorem]{Proposition}
\newtheorem{claim}[theorem]{Claim}
\newtheorem{corollary}[theorem]{Corollary}
\newtheorem{definition}[theorem]{Definition}
\newenvironment{proof}{{\bf Proof:}}{\hfill\rule{2mm}{2mm}}

% **** IF YOU WANT TO DEFINE ADDITIONAL MACROS FOR YOURSELF, PUT THEM HERE:

\newcommand\E{\mathbb{E}}

\begin{document}
%FILL IN THE RIGHT INFO.
%\lecture{**LECTURE-NUMBER**}{**DATE**}{**LECTURER**}{**SCRIBE**}
\lecture{4}{13 August}{Dr Dale Roberts}{Rui Qiu}
%\footnotetext{These notes are partially based on those of Nigel Mansell.}

% **** YOUR NOTES GO HERE:

% Some general latex examples and examples making use of the
% macros follow.  
%**** IN GENERAL, BE BRIEF. LONG SCRIBE NOTES, NO MATTER HOW WELL WRITTEN,
%**** ARE NEVER READ BY ANYBODY.

\textbf{\underline{Tools for Integration}}

Working with complex numbers and the sophisticated machinery of \underline{contour integration} will be needed in this course.

This week we are going to look at this beautiful area of Mathematics.

These tools will be very useful for calculations of the form

$$\int f(x)dF_y(x)$$

where $F_y$ is the MP distribution, for example.\\

\underline{\textbf{Complex numbers and elementary functions}}

With $i^2:=-1$, a complex number is an expression of the form $z=x+iy$. We write $Re(z)=x$ and $Im(z)=y$.

We can also write complex numbers in \underline{Polar form} $(r,\theta), x=r\cos\theta, y=r\sin\theta, r\geq 0$.

A complex number $z$ can be written

\begin{equation}
	z=x+iy=r(cos\theta+i\sin\theta)
\end{equation}

The \underline{radius} $r=\sqrt{x^2+y^2}=\lvert z\rvert$ (aka. \underline{modolus} of $z$) and the angle $\theta$ is the \underline{argument} of $z$, denoted $\arg z$.

When $z\not=0$, we can find $\theta$ by trigonometry

$$\tan\theta=\frac{y}{x}.$$

$\theta=\arg z$ is \underline{multivalued} as $\tan\theta$ is a periodic function of $\theta$ with period $\pi$.\\

\underline{Example:} $z=-1+i,\lvert z\rvert = r=\sqrt{2}, \theta=\frac{3\pi}{4}+2n\pi,n=0,\pm1,\pm2,\dots$

We can define the (polar) \underline{exponential}.

$$\cos\theta + i\sin\theta =e^{i\theta}\implies \text{by (4.1)} z=r\cdot e^{i\theta}$$

Some beautiful formulas:

\begin{equation}
	\begin{split}
		e^{2\pi i}&=1, e^{\pi}=-1, \dots\\
		e^{i\theta_1}e^{i\theta_2}&=e^{i(\theta_1+\theta_2)}, (e^{i\theta})^m=e^{im\theta}, (e^{i\theta})^{1/n}=e^{i\theta/n}
	\end{split}
\end{equation}

The \underline{complex conjugate} of $z$ is $\mean{z}=x-iy=re^{-i\theta}$.

The usual rules apply:

\begin{equation}
	\begin{split}
		z_1\pm z_2 &= (x_1\pm x_2)+ i(y_1\pm y_2)\\
		z_1z_2 &= (x_1+iy_1)(x_2+iy_2)\\
		&=(x_1x_2-y_1y_2)+i(x_1y_2+x_2y_1)		
	\end{split}
\end{equation}

Notice that $z\mean{z}=\mean{z}z=(x+iy)(x-iy)=x^2+y^2=\lvert z\rvert ^2$

so that

\begin{equation}
	\begin{split}
		\frac{z_1}{z_2}=\frac{x_1+iy_1}{x_2+iy_2}&=\frac{(x_1+iy_1)(x_2-iy_2)}{(x_2+iy_2)(x_2-iy_2)}\\
			&=\frac{(x_1x_2+y_1y_2)+i(x_2y_1-x_1y_2)}{x_2^2+y_2^2}\\
			&=\frac{x_1x_2+y_1y_2}{x_2^2+y_2^2}+i\frac{(x_2y_1-x_1y_2)}{x_2^2+y_2^2}
	\end{split}
\end{equation}

We can define same \underline{elementary functions} of complex argument. The simplest is

$$f(z)=z^n, n=0,1,2,\dots,\ \text{``Power function"}$$

A \underline{polynomial} of order $n$:

\begin{equation}
	\begin{split}
		P_n(z)&=\sum^n_{j=0}a_jz^j=a_0+a_1z+a_2z^2+\cdots +a_nz^n\\
		&\text{where $a_j$ are complex numbers.}
	\end{split}
\end{equation}

A \underline{rational function}:

\begin{equation}
	\begin{split}
		R(z)=\frac{P_n(z)}{Q_m(z)}, P_n(z), Q_m(z)\text{ are polynomials}.
	\end{split}
\end{equation}

In general, the function $f(z)$ is complex-valued and can be written

$$f(z)=u(x,y)+iv(x,y)=Ref+ Imf.$$

\underline{Example:} $z^2=(x+iy)^2=x^2-y^2+i2xy$ where $u(x,y)=x^2-y^2, v(x,y)=2xy$.\\

We can define the \underline{exponential function}

$$e^z=e^{x+iy}=e^xe^{iy}$$

and it is easy to show $e^z=e^x(\cos x+i\sin y).$

	$e^{z_1+z_2}=e^{z_1}e^{z_2}, \left(e^z\right)^n=e^{nz},n=1,2,\dots$
	
	$\lvert e^z\rvert =\lvert e^z\rvert \lvert \cos y +i\sin y\rvert =e^x\sqrt{\cos^2 y +\sin^2 y}= e^x$
		
	$\overline{(e^z)}=e^{\mean{z}}=e^x(\cos y- i\sin y)$\\
	
\underline{\textbf{Trig functions:}}

\begin{equation}
	\begin{split}
		\begin{split}\sin z&=\frac{e^{iz}-e^{-iz}}{2i}\\
		\cos z &=\frac{e^{iz}+e^{-iz}}{2i}
		\end{split}
	\end{split}
\end{equation}


$$\tan z=\frac{\sin z}{\cos z}, \cot z=\frac{\cos z}{\sin z}, \sec z=\frac{1}{\cos z}, \csc z=\frac{1}{\sin z}.$$

\underline{\textbf{Hyperbolic functions:}}

$$\sinh z =\frac{e^z-e^{-z}}{2}, \cosh z =\frac{e^z+e^{-z}}{2}, \tanh z=\frac{\sinh z}{\cosh z},$$

Note that

\begin{equation}
	\begin{split}
		\sinh iz = i\sin z,&\ \sin iz =i\sinh z\\
		\cosh iz = \cos z,&\ \cos iz =cosh z\\
	\end{split}
\end{equation}

Most of the definitions could have been introduced through the concept of \underline{power series}.

The power series of $f(z)$ around the point $z=z_0$ is 

$$f(z)=\lim_{n\to \infty}\sum^n_{j=0}a_j(z-z_0)^j=\sum^\infty_{j=0}a_j(z-z_0)^j$$

where $a_j, z_0$ are constants.\\

Remember convergence only occurs within some radius, i.e., within some circle $\lvert z-z_0\rvert =R$

$$R=\lim_{n\to\infty}\left\lvert\frac{a_n}{a_{n+1}}\right\rvert$$

We have the power series representations:

$$e^z=\sum^\infty_{j=0}\frac{z^j}{j!},$$
$$\sin z=\sum^\infty_{j=1}\frac{(-1)^jz^{2j+1}}{(2j+1)!},$$
$$\cos z=\sum^\infty_{j=1}\frac{(-1)^jz^{2j}}{(2j)!},$$
$$\sinh z=\sum^\infty_{j=0}\frac{z^{2j+1}}{(2j+1)!},$$
$$\cosh z=\sum^\infty_{j=0}\frac{z^{2j}}{(2j)!}.$$

Let $f(z)$ be defined in some region $R$ containing the neighbourhood of a point $z_0$. The \underline{derivative} of $f(z)$ at $z=z_0$, denoted by $f'(z_0)$ or $\frac{d}{dz}f(z_0)$ is defined by

$$f'(z_0)=\lim_{\Delta z\to 0}\left(\frac{f(z_0+\Delta z)-f(z_0)}{\Delta z}\right).$$

provided the limit exists. Alternatively, writing $\Delta z = z- z_0$.

$$f'(z_0)=\lim_{z\to z_0}\frac{f(z)-f(z_0)}{z-z_0}.$$

\underline{Caution}: A continuous function is not necessarily differentiable as complex functions have a two-dimensional character.

For example, $f(z)=\mean{z}$.

$$\lim_{\Delta z\to 0}\frac{\overline{(z_0+\Delta z)}-\overline{z_0}}{\Delta z}=\lim_{\Delta z\to 0}\frac{\overline{\Delta z}}{\Delta z}=0$$

and a unique value for $c$ cannot be found!

\underline{NOT DIFFERENTIABLE!}\\

Differentiable complex functions are called \underline{analytic}.

If $f$ and $g$ have derivatives:

\begin{equation}
	\begin{split}
		&(f+g)'=f'+g',\ (fg)'=f'g+g'f\\
		&\left(\frac{f}{g}\right)'=(f'g-fg')/g^2 (g\not=0)
	\end{split}
\end{equation}

If $f'(g(z))$ and $g'(z)$ exist, then

$$[f(g(z))]'=f'(g(z))g'(z)$$

Since $\frac{(z+\Delta z)^n-z^n}{\Delta z}=nz^{n-1}+a_1z^{n-2}\Delta z + a_2z^{n-3}(\Delta z)^2+\cdots +(\Delta z)^n\to nz^{n-1}$ as $\Delta z\to 0$, where $a_j$ are binomial coefficients of $(a+b)^n$.

We have $\frac{d}{dz}(z^n)=nz^{n-1}.$

If follows(formally) that

$$\frac{d}{dz}\left(\sum^\infty_{n=0}a_nz^n\right)=\sum^\infty_{n=0}na_nz^{n-1}\ \text{inside radius of convergence}$$

Writing $f(z)=u(x,y)+iv(x,y),$

\begin{equation}
	\begin{split}
		f'(z)&=\lim_{\Delta x\to 0}\left(\frac{u(x+\Delta x, y)-u(x,y)}{\Delta x}+i\frac{v(x+\Delta x, y)-v(x,y)}{\Delta x}\right)\\
		&=u_x(x,y)+iv_x(x,y)\\
		u_x:&=\frac{\partial u}{\partial x}, v_x:=\frac{\partial v}{\partial x.}\\
		f'(z)&=\lim_{\Delta y\to 0}\left(\frac{u(x,y+\Delta y)-u(x,y)}{i\Delta y}+i\frac{v(x,y+\Delta y)-v(x,y)}{i\Delta y}\right)\\
		&=-iu_y(x,y)+v_y(x,y)
	\end{split}
\end{equation}

Hence, equating these two expressions we have the \underline{Cauchy-Riemann equations}:

$$u_x=v_y, v_x=-u_y$$

that are necessarily satisfied if $f(z)$ is differentiable.

\begin{theorem}
	The function $f(z)=u(x,y)+iv(x,y)$ is differentiable at a point $z=x+iy$ of a region in the complex plane if and only if $u_x, u_y, v_x, v_y$ are continuous and satisfy the Cauchy Riemann equations.
\end{theorem}

We shall now look at how to evaluate integrals of complex-valued functions along curves in the complex plane.

First, consider the case of the complex-valued function $f$ of the real variable $t$, on the interval $a\leq t\leq b$.

$$f(t)=u(t)+iv(t)$$

We say that $f$ is \underline{integrable} if $u\& v$ are integrable, then 

$$\int^b_af(t)dt=\int^b_au(t)dt+i\int^b_av(t)dt.$$

Usual rules apply:

$$\frac{d}{dt}\int^t_a f(t)dt=f(t)\ text{``fundamental theorem of calculus"}$$

and if $f'(t)$ is continuous, $\int^b_af'(t)dt=f(b)-f(a)$.

\underline{\textbf{Integration along a curve}}

A curve in $\mathbb{C}$ can be described by the \underline{parametrization}

$$z(t)=x(t)+iy(t), a\leq t\leq b.$$

A curve $c$ is called \underline{simple} if it does not intersect itself. It is called \underline{differentiable} curve if $z'(t)=x'(t)+iy'(t)$ is non-null.

A \underline{piecewise differentiable curve (or path)} is obtained by joining a finite number of differentiable curves.

Let $C$ be a path, we call it \underline{closed} if $z(a)=z(b)$.

A closed path is also called a \underline{contour}.

\underline{Example:} The unit circle in $\mathbb{C}$ is a contour and is parametrized by $z(t)=e^{it}, 0\leq t\leq 2\pi$.

The \underline{contour integral} of a piecewise continuous function on a \underline{smooth contour} (i.e. differentiable) is defined to be

$$\int_C f(z)dz:=\int^b_af(z(t))z'(t)dt.$$

Note that $dz\approx z'(t)dt$.\\

Usual properties apply:

$$\int_C[\alpha f(z)+\beta yg(z)]dz = \alpha\int_C f(z)dz + \beta \int_C g(z)dz.$$

If  we traverse the contour in the opposite direction (i.e. form $t=b$ to $t=a$) then this is denoted $-C$ and

$$\int_{-C} f(z)dz = -\int_C f(z)dz.$$

And if $C=C_1\cup C_2\cup C_3\cup \cdots \cup C_n$ ,then

$$\int_C f=\sum^n_{j=1} \int_{C_j} f.$$

From the fundamental theorem of calculus we get:

\begin{theorem}
	Suppose $F(z)$ is an analytic function such that $f(z)=F'(z)$ is continuous in a domain $\mathcal{D}$. Then for the contour $C$ lying in $\mathcal{D}$ with endpoints $z_1$ and $z_2$
	
	$$\int_C f(z)dz=F(z_2)-F(z_1).$$
\end{theorem}

\begin{proof}
	\begin{equation}
		\begin{split}
			\int_C f(z)dz&=\int_C F'(z)dz=\int^b_a F'(z(t))z'(t)dt\\
			&=\int^b_a \frac{d}{dt}[F(z(t))]dt= F(z(b))-F(z(a))=F(z_2)-F(z_1).\\
			&\text{Note:} z(a)=z_1, z(b)=z_2.
		\end{split}
	\end{equation}
\end{proof}

Q: What happens if $C$ is a \underline{closed contour}?

$$\oint_C f(z)dz = \int_C F'(z)dz=0$$

``$\oint$" denotes integration along a closed contour $C$.

Notice that this holds for \underline{any} closed contour $C$. So this integral is independent of the path.\\

\begin{theorem}
	Let $f(z)$ be analytic interior to and on a simple closed contour $C$. Then at any interior point $z$
	
	$$f(Z)=\frac{1}{2\pi i}\oint_C\frac{f(\xi)}{\xi -z}d\xi.$$
\end{theorem}

This is the \underline{Cauchy integral formula}.

``The function $f$ is completely determined by the points $z\in C$"

Further, we can also say something about all the derivatives of $f$.\\

\begin{theorem}
	If $f(z)$ is analytic interior to and on a simple closed contour $C$, then all the derivatives $f^{(k)}(z), k=1,2,\dots$ exist in the domain $\mathcal{D}$ interior to $C$ and
	
	$$f^{(k)}(z)=\frac{k!}{2\pi i}\oint_C\frac{f(\xi)}{(\xi-z)^{k+1}}d\xi.$$
	
\end{theorem}

If $f(z)$ is an analytic function, we can establish its \underline{Taylor series} on its domain $\mathcal{D}=\{z:\lvert z\rvert\leq R\}$:

$$f(z)=\sum^\infty_{j=0}b_jz^j,\ b_j=\frac{f^{(j)}(0)}{j!}.$$

\underline{Example:} $e^z=\sum^\infty_{j=0}\frac{z^j}{j!},\ \lvert z\rvert < \infty$\\

In many situations we encounter functions that are not analytic everywhere in $\mathbb{C}$. Typically, they are not analytic at a point, or in some region.

This means that Taylor series \underline{cannot} be applied.

Luckily, another series representation can sometimes be found in terms of positive and negative powers of $(z-z_0)$.

\begin{theorem}
	(Laurent Series) A function $f(z)$ analytic in an annulus $R_1\leq \lvert z-z_0\rvert \leq R_2$ may be represented by the expansion
	
	$$f(z)=\sum^\infty_{n=-\infty}c_n(z-z_0)^n$$
	
	in the region $R_1<R_a\leq \lvert z- z_0\rvert \leq R_b < R_2$ where 
	
	$$c_n:=\frac{1}{2\pi i}\oint_C\frac{f(z)}{(z-z_0)^{n+1}}dz$$
	
	and $C$ is any simple closed contour in region of analyticity enclosing $\lvert z-z_0\rvert < R_1$.
	
	So $f(z)=\sum^\infty_{n=-\infty}c_n(z-z_0)^n$ and $c_{-1}$ is called the \underline{residue} of $f$. The negative powers are called the \underline{principal part} of $f$.
	
	$$c_{-1}:=\frac{1}{2\pi i}\oint_C f(z)dz$$
\end{theorem}

We call a point $z_0$ an \underline{isolated singularity} of a function $f$ if the function is analytic in the punctured disk

$$D=\{z:0<\lvert z-z_0\rvert\leq R.\}$$

Three types of singular points exist:

\begin{itemize}
	\item A \underline{removable singularity point} is when the Laurent series at the point has no terms with negative power $n<0$.
	\item A \underline{pole of order $m$} is an isolated singularity point such that$$f(z)=\sum^\infty_{n=-m}a_n(z-z_0)^n,\ a_{-m}\not=0.$$
	\item An \underline{essential singularity point} is an isolated singularity point where the Laurent series has infinitely many terms with negative power $n<0$.
\end{itemize}

\begin{theorem}
	Let $f(z)$ be analytic inside and on a simple closed contour $C$, except for a finite number of isolated singular points $z_1,z_2,\dots, z_N$ located inside $C$. Then
	
	$$\oint f(z)dz=2\pi i\sum^N_{j=1}a_j$$
	
	where $a_j$ is the \underline{residue} of $f(z)$ at $z=z_j$, denoted by $a_j:=Res(f(z), z_j).$
\end{theorem}

% **** THIS ENDS THE EXAMPLES. DON'T DELETE THE FOLLOWING LINE:

\end{document}





